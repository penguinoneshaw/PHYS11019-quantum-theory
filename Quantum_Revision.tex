\documentclass[]{scrreprt}
\usepackage{graphicx}
\usepackage[arrowdel]{physics}
\usepackage[hidelinks]{hyperref}
\usepackage{nicefrac}
\usepackage{amsmath}
\usepackage{amssymb}
\usepackage{amsfonts}
\usepackage{relsize}
\usepackage{booktabs}
\KOMAoption{chapterprefix}{true}

\KOMAoptions{
paper=a4,
fontsize=12pt,
parskip=half-,
BCOR=.5cm,
pagesize=auto,
pagesize=pdftex,
headinclude=false,
footinclude=false,
DIV=12
}

\renewcommand*\raggedchapter{\centering}
\RedeclareSectionCommand[beforeskip=0pt,afterskip=2\baselineskip]{chapter}
\setkomafont{chapterprefix}{\normalsize\mdseries}

\renewcommand*{\chapterformat}{%
  \chapappifchapterprefix{\nobreakspace}\thechapter\autodot%
  \IfUsePrefixLine{%
    \par\nobreak\vspace{-\parskip}\vspace{-.6\baselineskip}%
    \rule{0.9\textwidth}{.2pt}%
  }{}%
}

\usepackage{mathtools}
\DeclarePairedDelimiter\ceil{\lceil}{\rceil}
\DeclarePairedDelimiter\floor{\lfloor}{\rfloor}
\DeclarePairedDelimiter\integerpart{[}{]}
\usepackage{mathrsfs}
\newcommand{\fourier}[1]{\ensuremath{\mathlarger{\mathcal{F}}\!\!\left[#1 \right]}}
\newcommand{\inversefourier}[1]{\ensuremath{\mathlarger{\mathcal{F}^{-1}}\!\!\left[#1 \right]}}
\newcommand{\laplace}[1]{\ensuremath{\mathlarger{L}\!\!\left[#1 \right]}}
\DeclareMathOperator{\erfc}{erfc}

\setkomafont{disposition}{\rmfamily\scshape\bfseries}


\usepackage[largesc]{newpxtext}
\usepackage{newpxmath}
\setkomafont{descriptionlabel}{\rmfamily\bfseries}
\linespread{1.05}
\setcounter{tocdepth}{1}
\usepackage[tikz]{mdframed}
\newmdenv[frametitle=Example,roundcorner=5pt]{example}
\newmdenv[frametitle=Note,roundcorner=5pt]{note}

\title{Quantum Theory}
\subtitle{Integrals. SO MANY INTEGRALS.}
\author{James Shaw}
\date{Winter 2017}
\renewcommand{\leq}{\leqslant}
\renewcommand{\geq}{\geqslant}
\renewcommand{\vec}{\underline}
\setcounter{tocdepth}{0}

%\usepackage[activate={true,nocompatibility},final,tracking=true,kerning=true,spacing=true,shrink=10]{microtype}
\usepackage[activate={true,nocompatibility},kerning=true,spacing=true,final]{microtype}

\begin{document}
\makeatletter
\begin{titlepage}
\begin{minipage}[b]{0.8\textwidth}
\usekomafont{disposition}
        {\Huge School of Physics\\ and Astronomy\\}
\end{minipage}
\hfill
\begin{minipage}[t]{40mm}
        \includegraphics[width=40mm]{crest.eps}
\end{minipage}
\vspace*{4cm}

{\centering
{\LARGE\rmfamily\scshape \@title\\}
{\Huge\usekomafont{title} \@subtitle\\}
\vspace{\fill}
{\Large \@author \\[1cm] \LARGE\bfseries \@date\\}
}
\end{titlepage}
\makeatother


\tableofcontents % prints Table of Contents

\part{Back to Basics --- Quantum Mechanics Style}
\chapter{Tiny Little Things Sort Of Moving Around Not Actually That Fast}
\section{Double Slit Experiment}

The double slit experiment demonstrates some of the most important points of quantum mechanics.
Feynman was a particular fan of it. We make $P_1 = \abs{\phi_1}^2$ the probability of a particle passing through the first slit and likewise for $P_2$.
Classically, we'd expect that $$P_{1\text{ or }2} = P_1 + P_2 = \abs{\phi_1}^2 + \abs{\phi_2}^2,$$ which is all well and good but isn't what happens when we approach the quantum world.

We define two states, $\ket{i}$ and $\ket{f}$, the initial and final states, and two intermediate states $\ket{1}, \ket{2} \in \mathcal{H}$ for each of the slits.
These are vectors in a Hilbert space, and as such are linear superposable and
\end{document}
