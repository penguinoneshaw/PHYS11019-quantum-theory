\documentclass[]{scrreprt}
\usepackage{graphicx}
\usepackage[arrowdel]{physics}
\usepackage[hidelinks]{hyperref}
\usepackage{nicefrac}
\usepackage{amsmath}
\usepackage{amssymb}
\usepackage{amsfonts}
\usepackage{relsize}
\usepackage{booktabs}
\usepackage{bbm}
\KOMAoption{chapterprefix}{true}

\KOMAoptions{
paper=a4,
fontsize=12pt,
parskip=half-,
BCOR=.5cm,
pagesize=auto,
pagesize=pdftex,
headinclude=false,
footinclude=false,
DIV=12
}

\renewcommand*\raggedchapter{\centering}
\RedeclareSectionCommand[beforeskip=0pt,afterskip=2\baselineskip]{chapter}
\setkomafont{chapterprefix}{\normalsize\mdseries}

\renewcommand*{\chapterformat}{%
  \chapappifchapterprefix{\nobreakspace}\thechapter\autodot%
  \IfUsePrefixLine{%
    \par\nobreak\vspace{-\parskip}\vspace{-.6\baselineskip}%
    \rule{0.9\textwidth}{.2pt}%
  }{}%
}

\usepackage{mathtools}
\DeclarePairedDelimiter\ceil{\lceil}{\rceil}
\DeclarePairedDelimiter\floor{\lfloor}{\rfloor}
\DeclarePairedDelimiter\integerpart{[}{]}
\usepackage{mathrsfs}
\newcommand{\fourier}[1]{\ensuremath{\mathlarger{\mathcal{F}}\!\!\left[#1 \right]}}
\newcommand{\inversefourier}[1]{\ensuremath{\mathlarger{\mathcal{F}^{-1}}\!\!\left[#1 \right]}}
\newcommand{\laplace}[1]{\ensuremath{\mathlarger{L}\!\!\left[#1 \right]}}
\DeclareMathOperator{\erfc}{erfc}

\setkomafont{disposition}{\rmfamily\scshape\bfseries}


\usepackage[largesc]{newpxtext}
\usepackage{newpxmath}
\setkomafont{descriptionlabel}{\rmfamily\bfseries}
\linespread{1.05}
\setcounter{tocdepth}{1}
\usepackage[tikz]{mdframed}
\newmdenv[frametitle=Example,roundcorner=5pt]{example}
\newmdenv[frametitle=Note,roundcorner=5pt]{note}

\title{Quantum Theory}
\subtitle{Integrals. SO MANY INTEGRALS.}
\author{James Shaw}
\date{Winter 2017}
\renewcommand{\leq}{\leqslant}
\renewcommand{\geq}{\geqslant}
\renewcommand{\vec}{\underline}
\setcounter{tocdepth}{0}

%\usepackage[activate={true,nocompatibility},final,tracking=true,kerning=true,spacing=true,shrink=10]{microtype}
\usepackage[activate={true,nocompatibility},kerning=true,spacing=true,final]{microtype}

\begin{document}
\makeatletter
\begin{titlepage}
\begin{minipage}[b]{0.8\textwidth}
\usekomafont{disposition}
        {\Huge School of Physics\\ and Astronomy\\}
\end{minipage}
\hfill
\begin{minipage}[t]{40mm}
        \includegraphics[width=40mm]{crest.eps}
\end{minipage}
\vspace*{4cm}

{\centering
{\LARGE\rmfamily\scshape \@title\\}
{\Huge\usekomafont{title} \@subtitle\\}
\vspace{\fill}
{\Large \@author \\[1cm] \LARGE\bfseries \@date\\}
}
\end{titlepage}
\makeatother


\tableofcontents % prints Table of Contents

\part{Back to Basics --- Quantum Mechanics Style}
\chapter{Tiny Little Things Sort Of Moving Around Not Actually That Fast\ldots{} Maybe}
\section{Double Slit Experiment}

The double slit experiment demonstrates some of the most important points of quantum mechanics.
Feynman was a particular fan of it. We make $P_1 = \abs{\phi_1}^2$ the probability of a particle passing through the first slit and likewise for $P_2$.
Classically, we'd expect that $$P_{1\text{ or }2} = P_1 + P_2 = \abs{\phi_1}^2 + \abs{\phi_2}^2,$$ which is all well and good but isn't what happens when we approach the quantum world.

We define two states, $\ket{i}$ and $\ket{f}$, the initial and final states, and two intermediate states $\ket{1}, \ket{2} \in \mathcal{H}$ for each of the slits.
These are vectors in a Hilbert space, and as such are linear superposable and have dual basis vectors $\bra{i}, \bra{f}, \bra{1}, \bra{2} \in \mathcal{H}^\ast$.
These define the normalisation ($\braket{\psi} = 1$) and are used to great effect elsewhere.

The probabilities in this formalism are defined by
\begin{align*}
  P(i \to f) &= \abs{\bra{f}\ket{i}}^2\\
   &= \abs{\bra{f}\ket{1}\bra{1}\ket{i} + \bra{f}\ket{2}\bra{2}\ket{i}}^2\\
  &= \abs{\bra{f}\ket{1}\bra{1}\ket{i}}^2 + \abs{\bra{f}\ket{2}\bra{2}\ket{i}} + 2 \Re(\bra{f}\ket{1}\bra{1}\ket{i}^\ast \bra{f}\ket{2}\bra{2}\ket{i})
\end{align*}
where the completeness of the Hilbert space has been used.
This asserts that \[ \hat{\mathbbm{1}} = \sum_{\text{states}\, k} \dyad{k}\] where \( \ket{k}\in\mathcal{H} \) are orthonormal basis vectors.
This is what is meant by ``inserting a complete set of states''.
Othonormality of discrete eigenvectors (eignfunctions, eigenstates\ldots) is defined by \[ \braket{n}{m} = \delta_{mn}\] and in the continuous limit is defined by \[\braket{x}{x^\prime} \delta(x-x^\prime).\]

\section{Operators and Observables}
An observable \(\hat\xi = \sum_n \xi_n \op{n}\) is an operator which satisfies the following properties:
\begin{enumerate}
  \item That there are states \(\ket{n}\) which are eigenstates of $\hat \xi$ such that $\xi_n \in \mathbb{R}$ are the eigenvalues.
  This comes from the spectral theorem and representation.
  \item It is hermitian, that is to say \[\mel{\phi}{\hat\xi^\dagger}{\psi} =\mel{\phi}{\hat\xi}{\psi}\] which requires that the eigenvalues $\xi_n$ are real.
  \item It is complex-linear, such that \[ \hat\xi(c_1 \ket{\psi_1} + c_2 \ket{\psi_2} ) = c_1\hat\xi \ket{\psi_1} + c_2\hat\xi \ket{\psi_2} \] for $c_i \in \mathbb{C}$. This comes from the definition of the Hilbert space.
  \item It commutes with other observables, such that if $\hat{\chi}$ is an observable, then $\commutator{\hat\chi}{\hat\xi}=0$.
\end{enumerate}

When we measure an observable, we get a collapse of the quantum states.
This is shown through the use of the \emph{projection} operator, \[\hat{P}_n = \ketbra{n}\] which has the effect of throwing away all other states in a composite state.
For example, \[ \hat{P}_1 (c_1\ket{1} + c_2\ket{2}) = c_1\ket{1}\braket{1} + c_2\ket{1}\bra{1}\ket{2} = c_1 \ket{1} \] which seems nice and obvious in the maths, but the physics behind it is weird.

Degeneracy also makes this weird, but fundamentally it's just adding in more sums until everything is nicely in one state.



\end{document}
