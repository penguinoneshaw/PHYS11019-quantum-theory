\documentclass[]{revision-notes}

\addbibresource{QT.bib}

\title{PHYS11019: Quantum Theory}
\subtitle{Integrals. SO MANY INTEGRALS.}
\author{James Shaw}
\date{Spring 2017}

\begin{document}
\makeatletter
\begin{titlepage}
\begin{minipage}[b]{0.8\textwidth}
\usekomafont{disposition}
        {\Huge School of Physics\\ and Astronomy\\}
\end{minipage}
\hfill
\begin{minipage}[t]{40mm}
        \includegraphics[width=40mm]{crest.eps}
\end{minipage}
\vspace*{4cm}

{\centering
{\LARGE\rmfamily\scshape \@title\\}
{\Huge\usekomafont{title} \@subtitle\\}
\vspace{\fill}
{\Large \@author \\[1cm] \LARGE\bfseries \@date\\}
}
\end{titlepage}
\makeatother


\tableofcontents % prints Table of Contents

\chapter{Back to Basics --- Quantum Mechanics Style}
\section{Double Slit Experiment}

The double slit experiment demonstrates some of the most important points of quantum mechanics.
Feynman was a particular fan of it. We make $P_1 = \abs{\phi_1}^2$ the probability of a particle passing through the first slit and likewise for $P_2$.
Classically, we'd expect that $$P_{1\text{ or }2} = P_1 + P_2 = \abs{\phi_1}^2 + \abs{\phi_2}^2,$$ which is all well and good but isn't what happens when we approach the quantum world.

We define two states, $\ket{i}$ and $\ket{f}$, the initial and final states, and two intermediate states $\ket{1}, \ket{2} \in \mathcal{H}$ for each of the slits.
These are vectors in a Hilbert space, and as such are linear superposable and have dual basis vectors $\bra{i}, \bra{f}, \bra{1}, \bra{2} \in \mathcal{H}^\ast$.
These define the normalisation ($\braket{\psi} = 1$) and are used to great effect elsewhere.

The probabilities in this formalism are defined by
\begin{align*}
  P(i \to f) &= \abs{\bra{f}\ket{i}}^2\\
   &= \abs{\bra{f}\ket{1}\bra{1}\ket{i} + \bra{f}\ket{2}\bra{2}\ket{i}}^2\\
  &= \abs{\bra{f}\ket{1}\bra{1}\ket{i}}^2 + \abs{\bra{f}\ket{2}\bra{2}\ket{i}} + 2 \Re(\bra{f}\ket{1}\bra{1}\ket{i}^\ast \bra{f}\ket{2}\bra{2}\ket{i})
\end{align*}
where the completeness of the Hilbert space has been used.
This asserts that \[ \hat{\mathbbm{1}} = \sum_{\text{states}\, k} \dyad{k}\] where \( \ket{k}\in\mathcal{H} \) are orthonormal basis vectors.
This is what is meant by ``inserting a complete set of states''.
Othonormality of discrete eigenvectors (eignfunctions, eigenstates\ldots) is defined by \[ \braket{n}{m} = \delta_{mn}\] and in the continuous limit is defined by \[\braket{x}{x^\prime} \delta(x-x^\prime).\]

We can change the basis of a set of states with \[\ket{\bar{n}} = \sum_{m} \ket{m}\braket{m}{\bar{n}} \] where \(\braket{m}{n} = U_{mn}\) where $U \in \mathrm{SL}(n)$.

\section{Operators and Observables}
An observable \(\hat\xi = \sum_n \xi_n \op{n}\) is an operator which satisfies the following properties:
\begin{enumerate}
  \item That there are states \(\ket{n}\) which are eigenstates of $\hat \xi$ such that $\xi_n \in \mathbb{R}$ are the eigenvalues.
  This comes from the spectral theorem and representation.
  \item It is hermitian, that is to say \[\mel{\phi}{\hat\xi^\dagger}{\psi} =\mel{\phi}{\hat\xi}{\psi}\] which requires that the eigenvalues $\xi_n$ are real by the spectral theorem.
  \item It is complex-linear, such that \[ \hat\xi(c_1 \ket{\psi_1} + c_2 \ket{\psi_2} ) = c_1\hat\xi \ket{\psi_1} + c_2\hat\xi \ket{\psi_2} \] for $c_i \in \mathbb{C}$. This comes from the definition of the Hilbert space.
  \item It commutes with other observables, such that if $\hat{\chi}$ is an observable, then $\commutator{\hat\chi}{\hat\xi}=0$.
\end{enumerate}

When we measure an observable, we get a collapse of the quantum states.
This is shown through the use of the \emph{projection} operator, \[\hat{P}_n = \ketbra{n}\] which has the effect of throwing away all other states in a composite state.
For example, \[ \hat{P}_1 (c_1\ket{1} + c_2\ket{2}) = c_1\ket{1}\braket{1} + c_2\ket{1}\bra{1}\ket{2} = c_1 \ket{1} \] which seems nice and obvious in the maths, but the physics behind it is \emph{weird}.

Degeneracy also makes this weird, but fundamentally it's just adding in more sums until everything is nicely in one state.

\section{Squish It All Together}
We move to the continuous variables of position and momentum space.
This is equivalent to having infinite slits in your diffraction grating.

\subsection{Position}
Position space is defined such that \[ \int_a^b \psi(x) \ket{x} \dd{x}\] with the orthonormality condition \[\braket{x}{x^\prime}=\delta(x-x^\prime). \]
This means we can define the projection on to position space as a function of the position, i.e.~\[\braket{x}{\psi} = \psi(x).\]
The position operator is then \[ \hat{x} = \int x \ketbra{x} \dd{x} \] which is also known as the spectral resolution, and we've used the continuum analogue of the identity operator
\[ \hat{\mathbbm{1}} = \int \ketbra{x}\dd{x}. \]

\subsection{Momentum}
We use the fourier transform to find the momentum space.
We define states \(\ket{k}\) such that \[ \ket{k} = \frac{1}{\sqrt{2\pi}} \int e^{ikx} \ket{k} \dd{x}\] which allows us to make the identifications
\begin{align*}
  \braket{x}{k} &= \frac{1}{\sqrt{2\pi}}e^{ikx} & \braket{k}{x} &= \frac{1}{\sqrt{2\pi}}e^{-ikx}
\end{align*}
which shows that it is a unitary transformation between bases.

It's worth noting that \[ \tilde\psi(k) = \mathcal{F}\qty[\psi(x)].\]

\subsection{Working Between The Two}
We use Feynman's trick to find the action of the momentum operator on the position basis, given by
\begin{align*}
  \hat{k} \ket{x} &= \frac{1}{\sqrt{2\pi}} \int k e^{-ikx} \ket{k} \dd{k}\\
  &= \frac{i}{\sqrt{2\pi}} \pdv{x} \int e^{-ikx} \ket{k} \dd{k}\\
  &= i \pdv{x} \ket{x}
\end{align*}
which gives us the final results
\begin{align*}
  \mel{x}{\hat{k}}{\psi} &= -i\pdv{x} \psi & \mel{k}{\hat{x}}{\tilde\psi} &= i\pdv{k}\tilde\psi.
\end{align*}

\subsection{Canonical Commutation Relation}
The fundamental result of quantum mechanics is given by the canonical commutation relation, \[ \commutator{\hat{x}}{\hat{k}} = i \] which more practically is used as \[\commutator{\hat{x}}{\hat{p}} = i\hbar. \]

\section{Time is a Continuum. Lunchtime, doubly so}
We now increase the number of gratings, and take it to its continuum limit.
Firstly, we define a system with \(N\) gratings arranged one after another.
This gives us states \( \ket{x_i, t_i} \) where the \(i\) labels the grating, so \(t_i\) is the time the \(i\)-th grating is passed.
Using the Hilbert completeness relation, this gives us for \(N=1\) \[\braket{f}{i} = \int \braket{f}{x_1,t_1} \braket{x_1,t_1}{i}\dd{x_1}.\]
Increasing \(N\) further gives us \[\braket{f}{i} = \idotsint \braket{f}{x_1,t_1} \braket{x_1,t_1}{x_2,t_2} \bra{x_2, t_2}\cdots\ket{x_n,t_n}\braket{x_n, t_n}{f}\dd{x_1}\cdots\dd{x_n}\] where the increasing number of integrals increases the precision, but makes actual calculations difficult.

Defining \(\ket{i} = \ket{x_0, t_0} \) and \( \ket{f} = \ket{x_{N+1}, t_{N+1}} \) allows us to evenly space the gratings a distance \( \varepsilon \) from each other with
\begin{alignat*}{3}
  t_n &= t_0 + n\varepsilon & \qquad \text{where}&\qquad& \varepsilon &= \frac{t_{N+1} - t_0}{N+1}
\end{alignat*}
which then allows us to write the transition amplitude as
\begin{align*}
  \braket{x_b, t_b}{x_a, t_a} &= \qty(\prod_{n=1}^{N}\int\dd{x_n})\qty(\prod_{n=1}^{N+1}\braket{x_{n}, t_{n}}{x_{n-1}, t_{n-1}}).
\end{align*}

Finally, we define a new measure, which compactifies this notation for the \(\varepsilon \to 0 \) limit.
This is \[ \braket{x_a, t_a}{x_b, t_b} =  \eval{\int_{x_a}^{x_b}\DD{x}\braket{x_a, t_a}{x_b, t_b}}_{x(t)}. \]

\chapter{Tiny Little Things Sort Of Moving Around Not Actually That Fast\ldots{} Maybe}
\section{Classically Speaking, Actions Are Awkward}
The Lagrangian definition of the action is \[ S[x(t)] = \int_{t_a}^{t_b} L(x, \dot{x}, t) \dd{t} \]
where \(L = T-V \) is the classical Lagrangian (of a point particle in some generalised co-ordinate) and \(S[x(t)]\) is a functional of the position. We also have canonical momenta \[ p = \pdv{L}{\dot{x}} \] which coincide with regular momenta in most cases, rather like the energy function.

We define the functional derivative
\[ \eval{\frac{\delta S}{\delta x}}_{x = \bar x} = 0 \] which is a statement of Hamilton's principle, defining the classical trajectory where the endpoints are fixed. What it actually \emph{means} is that we vary the action along a path by an infinitesimal using \[ \delta S = S[x + \delta x] - S[x] \] and Taylor expand. This leads to the Euler-Lagrange equation for the classical trajectory, \[ \dv{t}(\pdv{L}{\dot x}) - \pdv{L}{x} = 0 \] which defines the classical trajectory \(\bar{x} \) on which the \emph{classical action}
\(S_{\mathrm{cl}} = S[\bar x(t)]\) is defined. This will come back later.

\section{Here Come The Irish With Their Principle Functions}
If we don't fix the later-in-time endpoint in space, we derive Hamilton's principle function.
The variation of \( (x_b, t_b) \mapsto (x_b, t_b) + (\delta x_b, \delta t_b)\) (where we then take \(\variation{t_b} = 0\)) gives us that this variation is \begin{align*}
  \variation{S_{\mathrm{cl}}} &= \eval{\pdv{L}{\dot{x}}\variation{x}}^{t_b}\\
  &= p(t_b)\variation{x}\\
  \implies \pdv{S_{\mathrm{cl}}}{x} &= p_b.
\end{align*}

Doing the equivalent variation for variation-in-time for the endpoint in space, we get the Hamilton-Jacobi equation for the Hamiltonian, \[\variation{S_{\mathrm{cl}} = - E_b \variation{t_b}}\] or otherwise written,
\[E(x_b, \pdv{S_{\mathrm{cl}}}{x_b}, t_b) + \pdv{S_{\mathrm{cl}}}{t_b} = 0\] which \emph{defines} Hamilton's principle function, which we've suspiciously also named \(S_{\mathrm{cl}}\).

\section{Just how probable are these paths, then?}
We go back to the transition amplitudes.
Through some slightly iffy arguments involving ``intuition'', we define the path amplitudes as
\begin{align*}
  \eval{\braket{x_a, t_a}{x_b, t_b}}_{x(t)} &\sim \exp[i\int_{t_a}^{t_b} \!\!\!\phi(x, \dot x, t) \dd{t}]\\
  &= e^{\nicefrac{i}{\hbar} S[x(t)]}.
\end{align*}
which we can get an expression for the full transition with
\begin{align*}
  \braket{x_b, t_b}{x_a, t_a} &= \int_{x_a}^{x_b}\!\DD{x} e^{\nicefrac{i}{\hbar} S[x(t)]}.
\end{align*}

Now, this still has the funny \(\DD{x}\) thing, which is defined as
\[\int_{x_a}^{x_b}\!\DD{x} = \lim_{N\to \infty} A_N \prod_{n=1}^N \int_{-\infty}^{\infty} \dd{x_{n}} \] where \(A_N = (\nu(\varepsilon))^{N+1}\) is a normalisation factor equal across each discrete step.
Ideally, it should make the entire thing converge, but we now need to find what \(\nu(\varepsilon)\) is.

\section{Where's a potential when you need one?}
We take a discrete time approximation for the Lagrangian of a free particle, that is to say
\[ L = T = \frac{1}{2} m \dot{x}^2 \approx \frac{1}{2} m \qty(\frac{x_1 - x_0}{\varepsilon})^2.\]
Turning this into a transition amplitude gives us (remembering that we've discretized time again)
\begin{align*}
 \braket{x_b, t_b}{x_a, t_a} &= \lim_{N\to \infty} A_N \prod_{n=1}^N \int_{-\infty}^{\infty} \dd{x_{n}} \exp[\frac{i\varepsilon}{\hbar} \frac{m}{2} \sum_{n=1}^{N+1}\qty(\frac{x_n - x_{n-1}}{\varepsilon})^2] \intertext{which, using results from \autoref{sec:nested-gaussians}, gives us}
 \braket{x_b, t_b}{x_a, t_a} &= \lim_{N\to \infty} A_N \qty(\frac{2 \varepsilon \hbar}{m})^{\nicefrac{N}{2}} \prod_{n=1}^N \int_{-\infty}^{\infty} \dd{x^\prime_{n}} \exp[i \sum_{n=1}^{N+1}\qty(\frac{x^\prime_n - x^\prime_{n-1}}{\varepsilon})^2]\\
 &= \lim_{N\to \infty} A_N \qty(\frac{2\pi i \varepsilon \hbar}{m})^{\nicefrac{N}{2}} \frac{1}{\sqrt{N+1}}\exp[\frac{i\qty(x^\prime_b - x^\prime_{a})^2}{\varepsilon(N+1)}]\\
 &= \lim_{N\to\infty}  A_N \qty(\frac{2\pi i \varepsilon \hbar}{m})^{\nicefrac{N+1}{2}} \sqrt{\frac{m}{2\pi i \hbar (N+1)\varepsilon}}\exp[\frac{im}{2\hbar}\frac{\qty(x_b - x_{a})^2}{\varepsilon(N+1)}].
\end{align*}
Now we must remember what we were trying to do --- find the normalisation \(\nu(\varepsilon)\) for the period \(T = t_b - t_a = (N+1)\varepsilon \).
To do this, we impose the requirement that
\[ A_N \qty(\frac{2\pi i \varepsilon \hbar}{m})^{\nicefrac{N+1}{2}} = 1\]
 which is equivalent to
 \[\nu(\varepsilon) = \sqrt{\frac{2\pi i \varepsilon \hbar}{m}},\]
  allowing us to simplify the above result into
\begin{equation}\label{eqn:free-particle}
  \braket{x_b, t_b}{x_a, t_a} = \sqrt{\frac{m}{2\pi i \hbar T}}\exp[i\frac{m}{2\hbar}\frac{\qty(x_b - x_{a})^2}{T}].
\end{equation}

Now, it can be worked out easily that for a free particle,
\[ S_{\mathrm{cl}} = \frac{m}{2} \frac{(x_b - x_a)^2}{T}\] which is suspiciously similar to \autoref{eqn:free-particle}, so we make an identification that
\begin{alignat*}{3}
  \braket{x_b, t_b}{x_a, t_a} &= F_0(T)e^{i\nicefrac{S_{\mathrm{cl}}}{\hbar}} & \qq{where} && F_0(T) &= \sqrt{\frac{m}{2\pi i \hbar T}}
\end{alignat*}
which is a \emph{very} helpful notation, since the prefactor is now independent of the endpoints.

You may notice that \( \nu(\varepsilon) \) diverges for infinitesimal \(\varepsilon \), which is a little inconvenient, but we basically ignore it, as that choice of \(\nu\) is only the first order term.

\section{Oscillation of its own accord}
For a harmonic oscillator, the ``discretize time'' method becomes awkward to do; however, there's another thing we can try.
Firstly, the Lagrangian for the harmonic oscillator is
\[ L = \frac{1}{2} m \dot{x}^2 - \frac{1}{2}m \omega x^2\] from which we can calculate the classical action,
\[ S_{\mathrm{cl}} = \frac{m\omega}{2\sin(\omega T)} [(x_a^2 + x_b^2)\cos(\omega T) - 2x_a x_b]\]
and the Newtonian equation of motion, \[ \ddot{x} + \omega^2 x = 0. \]

We will write \( x = \bar{x} + \eta \) where \(\eta(t_a) = \eta(t_b) = 0\) is the deviation from the classical path (allowing us to use the variational principle).
We find that
\begin{align*}
  S[x] = S[\bar{x} + \eta] &= \frac{m}{2}\int_{t_a}^{t_b} \qty[ \qty(\dot{\bar{x}} + \dot\eta)^2 - \omega^2(\bar{x} + \eta)^2] \dd{t}\\
                    &= S[\bar{x}] + S[\eta] + m\int_{t_a}^{t_b} \dot{\bar{x}}\dot\eta - \omega^2\bar{x}\eta\dd{t}\\
                    &= S[\bar{x}] + S[\eta] + \cancel{m[\eta \dot{\bar{x}}]_{t_a}^{t_b}} - m \int_{t_a}^{t_b}\eta \cancel{(\ddot{\bar{x}} + \omega^2 \bar{x}) }\dd{t}\\
                    &= S[\bar{x}] + S[\eta]
\end{align*}
which is, in fact, true for any Lagrangian quadratic in \(x\).
We can also argue that \[ \int_{x_a}^{x_b} \DD{x} = \int_{0}^{0} \DD{\eta}\] since \(\dd{x} = \dd{\eta} \), which looks a bit weird, but it's a useful statement.
This gives us our transition amplitude
\begin{align*}
  \braket{x_b, t_b}{x_a, t_a} &= \int_{x_a}^{x_b} e^{\nicefrac{i}{\hbar} S[x(t)]} \DD{x}\\
  &= e^{\nicefrac{i}{\hbar} S_{\mathrm{cl}}} \int_{0}^{0} e^{\nicefrac{i}{\hbar} S[\eta]} \DD{\eta}\\
  &\equiv F_\omega(T)e^{\nicefrac{i}{\hbar} S_{\mathrm{cl}}}
\end{align*}
which people solve in various ways, such as in \cite{feynman2010quantum}. Apparently the easiest is to do it by inspection, where we notice that since \( F_\omega (T) \) has no position dependence, we can also write it as
\begin{align*}
  F_\omega(T) &= \braket{0, T}{0, 0}\\
              &= \int_\mathbb{R} \bra{0,T}\ket{x,t}\bra{x,t}\ket{0,0} \dd{x}\\
              &= \int_\mathbb{R} \dd{x} F_\omega(T - t)e^{\nicefrac{i}{\hbar} S_{\mathrm{cl}}(T-t)}\,F_\omega(t)e^{\nicefrac{i}{\hbar} S_{\mathrm{cl}}(t)}\\
              &= F_\omega(T - t)\, F_\omega(t) \int_\mathbb{R} \dd{x} e^{\nicefrac{i}{\hbar} (S_{\mathrm{cl}}(T-t) + S_{\mathrm{cl}}(t))}
\end{align*}
at which point we use the definition of the classical action from earlier, notice that it's quadratic in \(x \) making it all a Fresnel integral, and abuse the nested Gaussians formula from \autoref{sec:nested-gaussians}.
This gives us
\begin{equation*}
  F_\omega(T) = F_\omega(T - t)\, F_\omega(t) \sqrt{\frac{2\pi i \hbar}{m\omega} \frac{\sin\omega(T-t) \sin\omega t}{\sin \omega T}}
\end{equation*}
which we can look at and notice that \( F_\omega (T) \) should probably be
\[ F_\omega (T) \sqrt{\frac{m\omega}{2\pi i \hbar} \frac{1}{\sin \omega T}} \] if we want the two other terms to cancel out. There's no systematic way of doing this really, though taking the \( T \gg t \) limit should result in the same.

\section{Driven to Distraction}
The forced harmonic oscillator has the Lagrangian
\[ L = \frac{1}{2} m (\dot{x}^2 - \omega^2 x^2) + J(t)x \]
and equation of motion for the classical path
\[ \ddot{\bar{x}} + \omega^2 \bar{x} = \frac{J}{m}. \]

This leaves the classical action \(S_\mathrm{cl}[\bar{x} + \eta](J) \), which helfully splits into independent terms like the unforced case, so it all works out rather nicely and we get
\begin{equation*}
  \braket{x_b, t_b}{x_a, t_a} = F_\omega(T)e^{\frac{i}{\hbar} S_\mathrm{cl}[\bar{x}](J)}
\end{equation*}
where we must calculate \(S_\mathrm{cl}\).

\section{Shiftying Everything in Time on Our Terms}
We look at a generic Lagrangian, \[ L= \frac{1}{2} m \dot{x}^2 - V(x, t) \] which we can Legendre transform into the Hamiltonian formalism and do the path integral to get
\[ \braket{x_b, t_b}{x_a, t_a} = \int \DD{x} \int \DD{p} \exp[\frac{i}{\hbar} \int_{t_a}^{t_b} \dd{t} (p\dot{x} - H(x, p, t))] \]
which comes from the infinitesimal amplitude
\begin{align*}
  \braket{x_{n+1}, t_{n+1}}{x_n, t_n} &= \sqrt{\frac{m}{2\pi i \varepsilon\hbar}} \exp[\frac{i\varepsilon}{\hbar}\qty{\frac{m}{2}\qty(\frac{x_{n+1} - x_n}{\varepsilon})^2 - V(x_n, t_n)}] \\
  &= \int_{-\infty}^\infty \frac{1}{2\pi\hbar}\exp[\frac{ip_n}{\hbar}(x_{n+1} - x_n) - \frac{i\varepsilon}{2m\hbar}p_n^2 - \frac{i\varepsilon}{\hbar}V(x_n, t_n)]\dd{p}\\
  &= \int_{-\infty}^\infty \mel{x_{n+1}}{\exp[-\frac{i\varepsilon}{2m\hbar}p_n^2]}{p_n} \mel{p_n}{\exp[ -\frac{i\varepsilon}{\hbar}V(x_n, t_n)]}{x_{n}}\dd{p}\\
  &=\mel{x_{n+1}}{\exp[-\frac{i\varepsilon}{2m\hbar}\hat{p}_n^2]\exp[ -\frac{i\varepsilon}{\hbar}V(\hat{x}_n, t_n)]}{x_{n}}
\end{align*}
which has successfully been moved into an operators form, and looks rather suggestive.
We can use the Baker-Campbell-Hausdorff formula \[ e^{\hat{A}}e^{\hat{B}} = e^{\hat{A} + \hat{B} + \frac{1}{2}\commutator{\hat{A}}{\hat{B}}} + \cdots \]
to first order to turn this into the infinitesimal time evolution operator \[\braket{x_{n+1}, t_{n+1}}{x_{n}, t_{n}} = \mel{x_{n+1}}{e^{-\frac{i\varepsilon}{\hbar{}} \hat{H}(t_n)}}{x_{n}}.\]
This \emph{defines} the time evolution operator for finite time for a time independent Hamiltonian,
\[ \ket{x, t} = \hat{U}(t, t_0)\ket{x} = e^{-\frac{i}{\hbar{}} (t - t_0) \hat{H}(t)}\ket{x}.\]

For time dependent Hamiltonians, we have (by the chain rule) that
\[ i\hbar \pdv{t}\hat{H}(t, t_0) = \hat{H}(t)\hat{U}(t, t_0). \]
\(\hat{U}\) is fundamentally changes in basis.

\section{Herren Schr\"odinger und Heisenberg kommen}
Using the time evolution operator, we can find derive the time-dependent Schr\"odinger equation, \[ i\hbar\pdv{t}\psi(x, t) = \hat{H}\psi(x, t) .\]

By defining a time dependent state vector \[ \ket{\psi, t} = \hat{U}(t, t_0) \ket{\psi} \] we introduce the Schr\"odinger picture of quantum mechanics where the bases are fixed and the states change, in opposition to the Heisenberg picture of fixed states and changing bases.
The choice of picture is a choice of where we put the time dependence, and does not change the physics of the problem.

In the Schr\"odinger picture, operators are time independent, so all evolution is done in-state. Contrastingly, in the Heisenberg picture, the operators are time \emph{dependent}, and are given by \[ \hat{x}(t) = \hat{U}^\dagger(t, t_0)\, \hat{x}\, \hat{U}(t, t_0) \] which is clearly the same if applied to a state. This time dependence gives the Heisenberg equation of motion almost immediately, which is
\begin{align*}
  i\hbar \pdv{t}\hat{O}(t) &= \commutator{\hat{O}(t)}{\hat{H}(t)}.
\end{align*}

We say that an operator is conserved if \(\pdv{\hat{O}}{t} = \commutator{\hat{O}(t)}{\hat{H}(t)} = 0 \).

\section{A British Miller (who happened to be really good at Maths) Cometh}
Moments of recognition might've triggered at some point that \( \braket{x_a, t_a}{x_b, t_b} \) is actually a Green's function of the Sch\"odinger equation, i.e.~
\begin{align*}
  (\hat{H} - i\hbar\pdv{t}) \braket{x, t}{x^\prime, t^\prime} &= -i\hbar \delta(t- t^\prime) \delta(x - x^\prime)
\end{align*}
although this isn't quite true, as the time parameter isn't fixed automatically.
Specifically, this is the retarded Green's function, where we must impose causality, giving us
\begin{equation*}
  G(x-x^\prime, t - t^\prime) =
  \begin{cases}
    \braket{x, t}{x^\prime, t^\prime} & t > t^\prime \\
    0 & t < t^\prime
  \end{cases}
\end{equation*}
which is kind of useful, since we can now use Green's function methods for finding states.

\section{Energy Eigenbasis}
We can use the eigenstates of the Hamiltonian as a pretty nifty basis, since the discretized energy levels means that everything is nicely sums. Specifically, if
\begin{align*}
  \hat{H}\ket{n} &= E_n\ket{n}\\
  u_n(x) &= \braket{x}{n}
\end{align*}
the practical upshot is that for \(t > 0 \), the colourful function is
\begin{align*}
  G(x-y, t) \equiv \braket{x, t}{y, 0} &= \mel{x}{e^{-\nicefrac{i}{\hbar}\,t \hat{H}}}{y}\\
  &= \sum_n e^{-\nicefrac{i}{\hbar}\,t E_n} u_n(x)\,u_n^\ast(y).
\end{align*}

The trace of the energy eigenbasis gives us \[ \int_{-\infty}^{\infty} \braket{x,t}{x,0} \dd{x} = \sum_n e^{-\nicefrac{itE_n}{\hbar}} \] which is really somewhat useful.\footnote{In the notes, at this point, there follows a good few examples which I simply cannot be bothered to type out, more's the pity. They don't actually seem to be that illuminating, though the Aharonov-Bohm effect is an interesting thing --- a magnetic field which by rights shouldn't effect the path of the particle does because integration.}

\section{Transition Elements}
The transition amplitude discovered several times now leads us to ask ``what is the weighted average of the paths on an operator?''.
We discover that it is
\begin{align*}
  \ev{x(t)}_S &\equiv \int_{x_a}^{x_b} x(t) \, e^{\frac{i}{\hbar} S[x]} \DD{x}\\
  &= \int_{-\infty}^\infty x \int_x^{x_b} \exp[\frac{i}{\hbar} \int_t^{t_b} L \dd{t^{\prime\prime}}]\DD{x^{\prime\prime}} \int^x_{x_a}\exp[\frac{i}{\hbar} \int^t_{t_a} L \dd{t^{\prime}}]\DD{x^{\prime}}\dd{x}\\
  &= \mel{x_b, t_b}{\hat{x}(t)}{x_a, t_b}
\end{align*}
which connects the path integrals to the traditional operator version.

Similarly, \[ \ev{f(x(t))}_S \equiv \int_{x_a}^{x_b} f(x(t))\, e^{\frac{i}{\hbar} S[x]} \DD{x} = \mel{x_b, t_b}{f(\hat{x}(t))}{x_a, t_b}\]
which is going to come in handy.

\section{Time Ordered Products}
It's worth noting, as it hasn't been stated previously, that positions at different times do not commute in the Heisenberg (or any) picture, that is to say \( \commutator{\hat{x}(t)}{\hat{x}(t^\prime)} \neq 0 \) for \( t \neq t^\prime \).
This motivates us to write
\[ \ev{x(t) x(t^\prime)}_S = \mel{x_b, t_b}{T(\hat{x}(t), \hat{x}(t^\prime))}{x_a, t_b} \]
where we define the \emph{time ordered product} as \[ T(\hat{x}(t), \hat{x}(t^\prime)) \equiv \theta(t - t^\prime)\hat{x}(t)\hat{x}(t^\prime) + \theta(t^\prime - t)\hat{x}(t^\prime)\hat{x}(t) \] which has the operators ordered from latest to earliest.

We can apply this to more general states \( \ket{\phi} \& \ket{\psi} \), giving
\begin{align*}
  \mel{\psi}{T(\hat{x}(t_1)) \cdots T(\hat{x}(t_2))}{\phi} &= \int \dd{x_a} \int \dd{x_b} \psi^\ast (x_b, t_b) \phi(x_a, t_a)  \ev{x(t_1)\cdots x(t_n)}_S
\end{align*}
which are also known as correlation functions.

\section{Another Austrian, Another Theorem, Classical Physics?}
By considering the spatial derivatives of the transition elements, we derive Ehrenfest's theorem,
\begin{equation*}
  \ev{\ddot{x}}_S = - \frac{1}{m} \ev{\pdv{V}{x}}_S
\end{equation*}
which looks a lot like Newton's 2nd law, and handily works like it too.

\chapter{Making It All A Bit Less Accurate}
\section{Perturbation Theory from Path Integrals}
Finding exact solutions is tricky (or impossible), and usually not worth it.
However, we can usually use perturbation theory by splitting the action into a solvable part and a perturbation, such as
\begin{align*}
  S[x(t)] &= S_0[x(t)] + S_1[x(t)]
\end{align*}
and then we can then consider (for slowly varying potentials), we can write
\begin{alignat*}{3}
  S_0[x(t)] &= \int_{t_a}^{t_b} \frac{1}{2}m\dot{x}^2 \dd{t} & \qq{and} &&  S_1[x(t)] &= -\int_{t_a}^{t_b} V(x(t), t) \dd{t}\intertext{or}
  S_0[x(t)] &= \int_{t_a}^{t_b} \frac{1}{2}m\dot{x}^2 - U(x) \dd{t} & \qq{and} &&  S_1[x(t)] &= -\int_{t_a}^{t_b} \tilde{V}(x(t), t) \dd{t}
\end{alignat*}
where we split \(V(x,t) = U(x) + \tilde{V}(x,t)\) where \(U\) is a solvable, time independent potential, often that of the Harmonic oscillator.

Substituting this separation into the transition amplitude gives
\begin{align*}
  \braket{x_b, t_b}{x_a, t_a} &= \int_{x_a}^{x_b} \exp[\frac{i}{\hbar}(S_0[x(t)] + S_1[x(t)])] \DD{x}\\
  &=  \int_{x_a}^{x_b} e^{\frac{i}{\hbar}S_0[x(t)]}\sum_{n=0}^\infty \frac{1}{n!}\qty(\frac{i}{\hbar}S_1[x(t)])^2 \DD{x}\\
  &= \braket{x_b, t_b}{x_a, t_a}_0 + \sum_{n=1}^\infty  \frac{1}{n!}\qty(\frac{i}{\hbar})^2 \prod_{k=1}^{n} \int_{t_a}^{t_b} \dd{t_n} \int_{x_a}^{x_b} \DD{x} \prod_{m=1}^{n}V(x(t_m), t_m) e^{\frac{i}{\hbar} S_0[x(t)] }
\end{align*}
where the last one uses the time ordered product.
Interpreting the last of those units is interesting, as
\begin{align*}
  \int_{x_a}^{x_b} \DD{x} \prod_{m=1}^{n}V(x(t_m), t_m) e^{\frac{i}{\hbar} S_0[x(t)] } = \int \dd{x_1} \cdots \int \dd{x_n} &\braket{x_b, t_b}{x_{n}, t_{n}}_0 V(x_n, t_n)  \\
  &\times \braket{x_{n}, t_{n}}{x_{n-1}, t_{n-1}}_0 \\
  & \times \cdots \times V(x_1, t_1) \braket{x_{2}, t_{2}}{x_{1}, t_{1}}_0
\end{align*}
which means that the perturbation can be thought of as the sum of situations where the particle is scattered through interaction with the potential an increasing number of times with each unit.

The \(x\) and \(t\) integrals are needed to sum over all paths (as is the basic conceit of the path integral) and the \( \frac{1}{n!}\) unit ensures that we don't overcount for each of the time ordered units.
There's an identity\footnote{which I genuinely don't see the use of} which is that
\begin{equation*}
  \frac{1}{n!}\int_{t_a}^{t_b} \dd{t_n} \cdots \int_{t_a}^{t_b} \dd{t_1} V(t_1) \cdots V(t_n) = \int_{t_a}^{t_b} \dd{t_n} \int_{t_a}^{t_n} \dd{t_{n-1}} \cdots \int_{t_a}^{t_2} \dd{t_1} V(t_1) \cdots V(t_n)
\end{equation*}
for\ldots reasons\ldots
Using this identity, we can get rid of the factorials and find that
\begin{align*}
    \braket{x_b, t_b}{x_a, t_a} &= \braket{x_b, t_b}{x_a, t_a}_0 + \sum_{n=1}^\infty \qty(-\frac{i}{\hbar})^n \prod_{k=1}^{n} \int_{t_a}^{t_{n+1}} \dd{t_n} \int_{x_a}^{x_b} \DD{x} \prod_{m=1}^{n}V(x(t_m), t_m) e^{\frac{i}{\hbar} S_0[x(t)] }.
\end{align*}

Some rewriting can occur, it all makes a meal out of getting to a statement which feels sort of intuitive, \begin{equation*}
  \braket{x_b, t_b}{x_a, t_a} = \braket{x_b, t_b}{x_a, t_a}_0  - \frac{i}{\hbar} \int_{-\infty}^\infty \dd{x} \int_{t_a}^{t_b} \dd{t} \braket{x_b, t_b}{x, t}_0 V(x, t) \braket{x, t}{x_a, t_a}.
\end{equation*}
This looks a little funky, but it's fine really; look at it as a weird application of the Green's function at different times.
It's worth noting that the last term (the \( \braket{x, t}{x_a, t_a}\)) is in relation to the full action, not just the unperturbed, so this is actually a hugely telescoping thing, which we can choose to use as many or as few terms of it as wanted.

We can apply this for a general wavefunction as
\begin{align*}
  \braket{x, t}{\psi} &= \braket{x, t}{\psi}_0 - \frac{i}{\hbar}\int_{-\infty}^\infty \dd{x^\prime} \int_{t_0}^t \dd{t} \braket{x, t}{x^\prime, t^\prime}_0 V(x^\prime, t^\prime) \braket{x^\prime, t^\prime}{\psi}
\end{align*}
which really emphasises the Green's function nature.

\section{Shooting Things At A Target Which Doesn't Move}
We invent a scattering operator, the \(S\) matrix, defined by
\begin{align*}
  \lim_{t_b \to \infty} \lim_{t_a \to -\infty} \braket{\vec{r}_b, t_b}{\vec{r}_a, t_a} &=   \lim_{t_b \to \infty} \lim_{t_a \to -\infty} \mel{\vec{r}_b}{\hat{U}(t_b, t_a)}{\vec{r}_a}\\
  &\equiv \mel{\vec{r}_b}{\hat{S}}{\vec{r}_a}\\
  &= \lim_{T \to \infty} \braket{\vec{r}_b, T}{\vec{r}_a, 0}
\end{align*}
which means we define in (at \( r_a \)) and out (at \( r_b \)) states at infinite time, but have the choice to start our internal clock whenever.

In the region outside the potential, we assume that the incident particle is a free particle with Lagrangian \[L_0 (\vec{r}) = \frac{m}{2} \qty(\dot{x}_1^2 + \dot{x}_2^2 + \dot{x}_3^2).\]

We then calculate to increasing orders in perturbation theory, the first order of which is
\begin{align*}
  A_1 &= - \frac{i}{\hbar} \int_{-\infty}^\infty \dd[3]{r} \int_{0}^{T} \dd{t} \braket{\vec{r}_b, t_b}{\vec{r}, t}_0 V(x, t) \braket{\vec{r}, t}{\vec{r}_a, t_a}\\
      &= - \frac{i}{\hbar} \int_{-\infty}^{\infty} \dd[3]{r} \int^T_0 \dd{t} \qty(\frac{m}{2\pi i \hbar (T-t)})^{\nicefrac{3}{2}}\exp{\frac{im\abs{\vec{r}_b - \vec{r}}^2}{2\hbar (T - t)}} V(\vec{r})\\
      &\mspace{300mu}\times\qty(\frac{m}{2\pi i \hbar t})^{\nicefrac{3}{2}}\exp{\frac{im\abs{\vec{r} - \vec{r}_a}^2}{2\hbar (T - t)}}.
\end{align*}
Solving that integral is a little tricky.
It requires a pretty complicated statement which you can probably just look up.
We set \( R_a = \abs{\vec{r}_a - \vec{r}} \) and \( R_b = \abs{\vec{r}_b - \vec{r}} \) and thus get
\begin{equation*}
A_1 = -\frac{i}{\hbar}\qty(\frac{m}{2\pi i\hbar{}})^{\nicefrac{5}{2}} \frac{1}{T^{\nicefrac{3}{2}}} \int^\infty_{-\infty} \qty(\frac{1}{R_a} + \frac{1}{R_b}) V(\vec{r}) \exp[\frac{im}{2\hbar T}(R_a + R_b)^2] \dd[3]{r}
\end{equation*}

We continue our quest to rewrite this ever less accurately by assuming that the potential is short range and thus expanding the statements
\[ R_i = \abs{\vec{r}_i - \vec{r}} \approx r_i - \vec{n}_i \vdot \vec{r} \]
which, with substitution and expansion, gives
\begin{multline*}
  A_1 = -\frac{i}{\hbar}\qty(\frac{m}{2\pi i\hbar{}})^{\nicefrac{5}{2}} \frac{1}{T^{\nicefrac{3}{2}}} \qty(\frac{1}{r_a} + \frac{1}{r_b}) \exp[\frac{im}{2\hbar T}(r_a + r_b)^2] \\
  \times\int^\infty_{-\infty}  V(\vec{r}) \exp[\frac{im}{\hbar T}(r_a + r_b)(\vec{n}_a + \vec{n}_b)\vdot{\vec{r}}] \dd[3]{r}
\end{multline*}

\subsection{Transfers}
By assuming that most of the time the particle is a free particle, we can assume its speed is \[ u = \frac{r_a + r_b}{T} \] which then makes
\begin{align*}
  p &= m\frac{r_a + r_b}{T} & E &= \frac{1}{2} m \qty(\frac{r_a + r_b}{T})^2
\end{align*}
and the vector momentum is thus \[ \vec{p}_i = -p \vec{n}_i \qq{for} i \in \{a, b\}. \]
The momentum transfer (i.e.~the change in momentum due to the interactions) is
\begin{align*}
  \vec{p}_a - \vec{p}_b &= -m\frac{r_a + r_b}{T} (\vec{n}_a + \vec{n}_b)\\
  &\equiv \hbar \vec{k}
\end{align*}

This gives us \emph{YET ANOTHER} rewriting,
\begin{align*}
  A_1 &= -\frac{i}{\hbar}\qty(\frac{m}{2\pi i\hbar{}})^{\nicefrac{5}{2}} \frac{1}{T^{\nicefrac{3}{2}}} \qty(\frac{r_a + r_b}{r_a r_b}) \exp[\frac{i}{\hbar}ET] \int^\infty_{-\infty}  V(\vec{r}) \exp[i\vec{k}\vdot{\vec{r}}] \dd[3]{r}
\end{align*}
and we define (for the sake of easier writing later)
\begin{align*}
  \tilde{V}(\vec{k}) &= \int^\infty_{-\infty}  V(\vec{r}) \exp[i\vec{k}\vdot{\vec{r}}] \dd[3]{r}
\end{align*}
which also emphasises that this is just the Fourier transform of the potential --- wave-particle duality is becoming a thing.

\subsection{Transition Probability}
The transition probability (from the starting position to a specific endpoint)
\begin{equation*}
  P(a \to b) = \abs{A_1}^2 = \frac{1}{\hbar^2} \qty(\frac{m}{2\pi\hbar})^5 \frac{1}{T^3} \qty(\frac{r_a + r_b}{r_a r_b})^2 \abs{\tilde{V}(\vec{k})}^2.
\end{equation*}

For a no-scatter situation --- where the particle doesn't interact with the potential --- we use the free particle transition amplitudes
\begin{align*}
  A_0 = \braket{\vec{r}_c, T}{\vec{r}_a, 0}_0 &= \qty(\frac{m}{2\pi i \hbar T})^{\nicefrac{3}{2}} \exp[i\frac{m}{2\hbar T}(\vec{r}_c - \vec{r}_a)^2]\\
  \implies P(a \to c) = \abs{A_0}^2 &= \qty(\frac{m}{2\pi i \hbar T})^{3}.
\end{align*}

Putting these together to get the transition ratio gives
\begin{align*}
  \frac{P(a \to b)}{P(a \to c)} &= \qty(\frac{m}{2\pi\hbar^2})^2 \qty(\frac{r_a + r_b}{r_a r_b})^2 \abs{\tilde{V}(\vec{k})}^2.
\end{align*}

\subsection{Cross Section}
The differential cross section (in the Born approximation) is the rate of scattering against an infinitesimal area \( \dd{\sigma}\) into an infinitesimal solid angle \( \dd{\Omega} \).
This gives us the statement that
\begin{align*}
  \frac{P(a \to b)}{P(a \to c)} &= \qty(\frac{r_a + r_b}{r_a r_b})^2 \dv{\sigma}{\Omega}\\
  \implies\qquad \dv{\sigma}{\Omega} &= \qty(\frac{m}{2\pi\hbar^2})^2 \abs{\tilde{V}(\vec{k})}^2.
\end{align*}

If \(V(\vec{r}) \) only depends on r (i.e.~is a central potential), then
\begin{align*}
  \tilde{V}(\vec{k}) &= \int_{-\infty}^{\infty} V(r) \exp i \vec{k}\vdot\vec{r} \\
  &= \frac{4\pi}{k}\int_0^\infty r\,V(r) \,\sin(kr)\dd{r}.
\end{align*}
In order to get this to converge, we will sometimes have to modify this to
\begin{align*}
  \tilde{V}(\vec{k}) &= \lim_{\mu \to 0} \int_0^\infty r\,V(r) \,\sin(kr) e^{-\mu r}\dd{r}
\end{align*}
where \( k = \frac{1}{\hbar} \abs{\vec{p}_a - \vec{p}_b} = \frac{2p}{\hbar}\sin(\nicefrac{\theta}{2}) \).

\begin{example}
  For any \(V(r) \propto -\frac{1}{r}\), we can use the forced-convergence trick to find that
  \begin{align*}
    \int_0^\infty r\,V(r)\,\sin(kr) \dd{r} &\propto -\frac{1}{k}
  \end{align*}
  and hence
  \begin{align*}
    \dv{\sigma}{\Omega} &\propto \frac{1}{16E^2}\frac{1}{\sin^4(\nicefrac{\theta}{2})}
  \end{align*}
  which, in the case of the Coulomb potential, happens to coincide with the Rutherford cross section.
\end{example}

\section{Shooting Beams at Each Other}
Two particles with masses \(m_1\) and \(m_2 \) interact with each other with a potential from either (effectivley pairwise).
The system has a two-variable Lagrangian
\[ L = \frac{1}{2}m_1 \abs{\pdv{\vec{r}_1}{t}}^2 + \frac{1}{2}m_2 \abs{\pdv{\vec{r}_2}{t}}^2 - V(\vec{r}_1 - \vec{r}_2). \]
We shuffle this into the centre-of-mass frame, defining the postition of the centre of mass,
\begin{align*}
  \vec{R} &= \frac{m_1 \vec{r}_1 + m_2 \vec{r}_2}{M} & M &= m_1 + m_2 \intertext{and relative motion within the system}
  \vec{r} &= \vec{r}_1 - \vec{r}_2 & \mu &= \frac{ m_1 m_2 }{m_1 + m_2}
\end{align*}
to give us a new Lagrangian
\[ L = \frac{1}{2} M \abs{\pdv{\vec{R}}{t}}^2 + \frac{1}{2}\mu \abs{\pdv{\vec{r}}{t}}^2 - V(\vec{r}). \]
In the centre-of-mass frame, \(\vec{R} = 0\), and \(L = \frac{1}{2}\mu \abs{\pdv{\vec{r}}{t}}^2 - V(\vec{r})\) which is the same as for the fixed-target, so we can use the same result as earlier, but define a new function
\begin{align*}
  \dv{\sigma}{\Omega} &= \qty(\frac{\mu}{2\pi\hbar^2})^2 \abs{\tilde{V}(\vec{k})}^2\\
  &\equiv \abs{f(\theta, \phi)}^2
\end{align*}
which we call the scattering amplitude.

We use the Born-Oppenheimer approximation where we assume one particle much more massive than the other, and this recovers the fixed-particle result.

If we have two indistinguishable particles, then there are two indistinguishable outcomes of the collision, resulting in
\begin{align*}
  \dv{\sigma}{\Omega} &= \abs{f(\theta) + f(\pi - \theta)}^2.
\end{align*}
This allows us to distinguish between fermions and bosons, since the result will be different due to the spin statistics theorem (fermions are antisymmetric).

\section{Operator?}
We can develop perturbation with the time evolution operator.
We define a new picture, the Dirac picture, which is the same as the Heisenberg picture but is done with respect to \( S_0 \) instead of \(S\), such that
\begin{equation*}
  i\hbar \pdv{t} \hat{U}_0 = \hat{H}_0 \hat{U}_0
\end{equation*}
for a time dependent Hamiltonian, and
\begin{equation*}
  \hat{U}_0 = \exp[\frac{i}{\hbar} (t - t_0) \hat{H}_0].
\end{equation*}
This allows us to define a new set of Heisenberg operators, for example
\[ \hat{x}_0 (t) = \hat{U}_0^\dagger (t, t_0) \, \hat{x} \,  \hat{U}_0(t, t_0)\]

\section{Time Dependent Transition}
Going all of the way back to the statement for the perturbative transition amplitude series, we can put the whole thing into the operator formalism, such that
\begin{align*}
  \hat{U}(t_a, t_b) &= \hat{U}_0 (t_b, t_a) + \sum_{n=1}^\infty (-\frac{i}{\hbar})^n \int^{t_b}_{t_a} \dd{t_n} \cdots \int^{t_2}_{t_a} \dd{t_1}\\
  &\mspace{100mu}\times \hat{U}_0(t_b, t_n) V(\hat{x}, t_n) \hat{U}_0(t_n, t_{n-1}) \cdots V(\hat{x}, t_1) \hat{U}_0(t_1, t_a)\\
  &= \hat{U}_0 (t_b, t_a) -\frac{i}{\hbar} \int^{t_b}_{t_a} \dd{t_n} \hat{U}_0(t_b, t) V(\hat{x}, t) \hat{U}(t, t_{a})\\
\end{align*}

Now we remember that in the energy basis, \(\hat{H}_0\) is a diagonal operator, which is useful, since \[ \braket{m,t}{n^\prime, t^\prime} = e^{-\frac{i}{\hbar}(t - t^\prime)E_n} \delta_{mn}.\]

Using this for the transition gives
\begin{align*}
  \braket{b, t_b}{a, t_a} &= \mel{b}{\hat{U}(t_b, t_a)}{a}\\
  &= \mel{b}{\hat{U}_0(t_b, t_a)}{a} - \frac{i}{\hbar} \int_{t_a}^{t_b} \mel{b}{\hat{U}(t_b, t)V(\hat{x}, t)\hat{U}(t, t_a)}{a} \dd{t} + \cdots\\
  &= e^{-\frac{i}{\hbar}(t_b - t_a)E_a}\delta_{ab} - e^{-\frac{i}{\hbar}(E_bt_b - E_at_a)} \int_{t_a}^{t_b} \dd{t} e^{\frac{i}{\hbar}t (E_b - E_a)}V_{ba}(t)+ \cdots
\end{align*}
which corresponds to more and more interactions you add more potential interactions at different times (as number of terms increases).

Using this statement, we can see the first order interaction probability is
\begin{align*}
   P(a \to b) &= \frac{1}{\hbar^2} \abs{\tilde{V}_{ba}}^2\intertext{where}
   \tilde{V}_{ba}(t) &= \int_{t_a}^{t_b} \dd{t}V_{ba}(t) \intertext{where we have defined by convention}
   V_{ab} &\equiv \mel{a}{V(\hat{x}, t)}{b} = \int_{-\infty}^\infty u_m^\ast(x) V(x, t) u_n(x).
\end{align*}

If the perturbation is time independent, we derive Fermi's golden rule for the transition rate, \[ R = \frac{2\pi}{\hbar}\abs{V_{ba}}^2 \rho(E_b) \] which is covered extensively elsewhere.

\chapter{Richard Feynman Sticks His Boot In}
We go back to classical mechanics for a bit, in order to motivate the point of the exercise.
A forced \emph{an}harmonic oscillator has an action \(S = S_0[x]_J\footnote{I'm not certain about this notation, but I know for sure and certain I don't like the one in the notes. It's also significantly different to the one I used earlier.} + S_1[x]\) with Lagrangian \[L = L_{\text{Driven SHO}} - \frac{\lambda}{4}x^4 \] and
\[ \qty(\pdv[2]{t} + \omega^2)x = \frac{J}{x} - \frac{\lambda}{m}x^3.\]


\section{GREEEEEEEN *waves fist*}
We have the homogeneous equation \[ \qty(\pdv[2]{t} + \omega^2) \bar{x}_0 = 0 \] with logical (Neumann) boundary conditions.
With this in mind, we build a Green's function \(\Delta(t, t^\prime)\) such that \[\qty(\pdv[2]{t} + \omega^2) \Delta(t, t^\prime) = -\delta(t - t^\prime) \] with boundary conditions \( \Delta(t_a, t^\prime) = \Delta(t_b, t^\prime) = 0 \).

We call \(\Delta\) the Feynman Propagator, and solve it as
\begin{multline*}
  \Delta(t, t^\prime) = \frac{1}{\omega \sin\omega T}\big[\theta(t - t^\prime)\sin\omega(t_b - t)\,\sin\omega(t^\prime-t_a) \\+ \theta(t^\prime - t)\sin\omega(t_b - t^\prime)\,\sin\omega(t-t_a)\big]
\end{multline*}
which means that (as usual) the full solution for the classical solution is
\begin{align*}
  \bar{x}(t) = \bar{x}_0(t) + \frac{1}{m}\int_{t_a}^{t_b} \Delta(t, t^\prime) \qty(-J(t^\prime) + \lambda \bar{x}(t^\prime)^3) \dd{t^\prime}.
\end{align*}
You may notice the similarities between this and the previously derived quantum perturbation theories.
This is intentional.
The equation maps out ``trees of four-point interactions'' somehow.

\section{And again, it all becomes wibbly and weird}
We take our forced harmonic oscillator and sort it out so that
\begin{align*}
  S_0[x]_J &= \int_{t_a}^{t_a} \qty(\frac{m}{2} (\dot{x}^2 - \omega^2 x^2) + Jx)\dd{t} & S_1[x] &= -\frac{\lambda}{4} \int_{t_a}^{t_b} x^4 \dd{\tau}.
\end{align*}
This makes our transition (using functional derivatives)
\begin{align*}
  \mel{x_b, t_b}{T(\hat{x}(t_1)\cdots\hat{x}(t_m))}{x_a, t_a}_J &= \int x(t_1)\cdots x(t_m)\,e^{\frac{i}{\hbar}S[x]_J} \DD{x}\\
  &= \int x(t_1)\cdots x(t_m)\,e^{\frac{i}{\hbar}S[x]_0 + \int_{t_a}^{t_b} J(t) x(t) \dd{t}} \DD{x}\\
  &= \qty(\frac{\hbar}{i})^m \frac{\delta^m}{\delta J(t_1)\,\cdots\,\delta J(t_m)} \braket{x_b, t_b}{x_a, t_a}_J.
\end{align*}

\subsection{Wibblier and Weirder}
We expand \(\exp[\frac{i}{\hbar} S_1[x]] = \exp[-\frac{i}{\hbar} \frac{ \lambda }{4}\int_{t_a}^{t_b} \dd{\tau} x(\tau)^4]\) to get a perturbation expansion for the anharmonic oscillator
\begin{align*}
  \braket{x_b, t_b}{x_a, t_a}_J &= \int_{x_a}^{x_b} \DD{x} \exp[\frac{i}{\hbar}(S_0[x]_J + S_1[x])]\\
  &= \sum_{n=0}^{\infty} \frac{1}{n!} \qty(-\frac{i\lambda}{4\hbar})^n \int_{t_a}^{t_b} \dd{\tau_1} \cdots \int_{t_a}^{t_b} \dd{\tau_n}  \int_{x_a}^{x_b} \DD{x} x(\tau_1)^4 \cdots x(\tau_2)^4 e^{\frac{i}{\hbar} S_0[x]_J} \\
  &= \sum_{n=0}^{\infty} \frac{1}{n!} \qty(-\frac{i\lambda}{4\hbar})^n \qty[\prod_{k=1}^n \int_{t_a}^{t_b} \dd{\tau_k} \qty{\frac{\hbar}{i}\fdv[k]{J(\tau_k)}}^4] \int_{x_a}^{x_b} \DD{x}  e^{\frac{i}{\hbar} S[x]_J}\\
  &= \exp[-\frac{i\lambda}{4\hbar} \qty(\frac{\hbar}{i}\fdv[k]{J(\tau_k)})^4] F_\omega(T) \exp[\frac{i}{\hbar} S[\bar{x}]_J].
\end{align*}

If we make the assumptions required to use the Feynman Propagator, we can find a general expression for the correlation functions
\begin{multline*}
  \frac{1}{F_\omega(T)} \mel{0, t_b}{T(\hat{x}(t_1)) \cdots T(\hat{x}(t_n))}{0, t_a} \\= \qty(\frac{\hbar}{i})^m \frac{\delta^m}{\delta J(t_1)\,\cdots\,\delta J(t_m)} \exp[-\frac{i\lambda}{4\hbar} \qty(\frac{\hbar}{i}\fdv[k]{J(\tau_k)})^4]\\\times \exp[\frac{i}{\hbar} \int_{t_a}^{t_b}\int_{t_a}^{t_b}J(t)\Delta(t, t^\prime)J(t^\prime)\dd{t^\prime}\dd{t}].
\end{multline*}

\section{Special Results}
We have Wick's Theorem for the \( n\)-point correlation function (time ordered product of \(n\) position operators) which is that for a pure forced simple harmonic oscillator,
\begin{align*}
  \frac{1}{F_\omega (T)} \mel{0, t_b}{T(\hat{x}(t_1) \cdots \hat{x}(t_n))}{0, t_a}_{J=0} &= \qty(\frac{i\hbar}{m})^{\nicefrac{m}{2}} \sum_{\text{all pairings}}\Delta(t_{i_1}, t_{i_2})\cdots \Delta(t_{i_{m-1}}, t_{i_m}).
\end{align*}

By considering the first order of the first exponential (the anharmonic term), we can derive the first of the Feynman rules.
It's worth noting that in the integrals, we only care about terms which keep a \( J(\tau) \) after functional differentiation, so the entire thing reduces to \[\frac{1}{F_\omega(T)} \mel{0, t_b}{T(\hat{x}(t_1) \hat{x}(t_2) \hat{x}(t_3) \hat{x}(t_4)) }{0, t_a}_{J=0} = -i \frac{3!\, \lambda}{\hbar}\int_{t_a}^{t_b} \dd{t} \prod_{i=1}^{4}(\frac{i\hbar}{m}\Delta(t, t_i))\]
which represents an interaction (or a crossing point on a Feynman diagram) since they all four propagators go terminate or start at \(t\).

\section{Doing It Over All Time}
None of these integrals converge for all times, since \(\Delta(t, t^\prime)\) doesn't have a limit and oscillates \emph{wildly}.
We have three choices in this case, which are all roughly equivalent:
\begin{enumerate}
  \item Take \( t = -i\tau \) as is the case in Statistical Mechanics\footnote{which is one of the examples which were skipped earlier}
  \item Rotate \(t\) in the complex plane, such that \( t \mapsto t - i\epsilon \)
  \item Dampen the path integral by mapping \[ S^\prime = S + i\epsilon \int_{-\infty}^\infty x^2 \dd{t} \] which is called the `\(i\epsilon\) prescription'. In this prescription we have \( (\omega^\prime)^2 = \omega^2 - i \epsilon \).
\end{enumerate}

For the propagator over all time, we define a new, cooler version which does everything the old one does and converges.
With all of the rewriting we've done, this feels like an anticlimax, but we find that
\begin{align*}
  \lim_{t_b\to \infty}\lim_{t_a\to -\infty}\Delta(t, t^\prime) &= \frac{1}{2i\omega}\qty(e^{-i\omega(t-t^\prime)} \theta(t-t^\prime) + e^{i\omega(t-t^\prime)} \theta(t^\prime - t))\\
  &= \Delta_F(t- t^\prime)
\end{align*}

\chapter{Lots of Little Things Moving Around Really Rather Quickly\ldots{} Maybe}
\section{The Klein-Gordon equation}
In special relativity\footnote{I've skipped the four-vector stuff}, we know that the energy of a particle is given by
\[ E^2 = p^2 c^2 + m^2 c^4 \]
which mean that if we use the standard substitutions (\(E \mapsto i\hbar \pdv{t}\) and \(\vec{p}\mapsto i\hbar\grad\)) which allows us to derive
\begin{align*}
  m^2 c^4 \phi(\vec{r}, t) - \hbar^2 c^2 \laplacian \phi(\vec{r}, t) &= -\hbar^2 \pdv[2]{t}\phi(\vec{r}, t)\\
  \implies \qquad (\hbar^2\partial^2 + m^2 c^2)\phi(\vec{r}, t) &= 0
\end{align*}
which is the Klein-Gordon equation for the free particle. This is solvable with a 4-vector plane wave expression \[\phi(x) = e^{-ik\vdot x} \equiv e^{-i\frac{p}{\hbar}\vdot x}.\]

This has a couple of problems.
Firstly, our basic idea that \(\hbar^2( \omega^2 - c^2 \vec{k}^2 ) = m^2 c^4\) implies that there are negative energy particles; we'll deal with that later.

Secondly, we can derive the probability current for the Schr\"odinger equation \[0 = \psi^\ast (\text{SE})  - \psi (\text{SE})^\ast \] which expands to
\[\pdv{\rho}{t} + \div\vec{j} = 0\]
where we find that
\begin{align*}
  \rho &= \psi^\ast \psi & \vec{j} &= -\frac{i\hbar}{2m}\qty(\psi^\ast \grad\psi - \psi \grad\psi^\ast)
\end{align*}
and for the Klein Gordon equation,
\begin{align*}
  \rho &= \frac{i\hbar}{2mc^2}\qty(phi^\ast \pdv{t}\phi - phi \pdv{t}\phi^\ast) & \vec{j} &= -\frac{i\hbar}{2m}\qty(\phi^\ast \grad\phi - \phi \grad\phi^\ast)
\end{align*}
which has the issue that \(\rho\), the probability density, is not positive definite, so we can't interpret it \emph{as} a probability density.

\section{Yet another thing named after Dirac}
To combat the two negative things, we try to find a different relativistic wave equation which is linear in \( \pdv{t}\).
We start with the Schr\"odinger equation again, in the form
\[ i\hbar\pdv{t}\psi(x) = \hat{H}\psi(x) \]
which we make a reasonably radical ansatz for, and get
\begin{align*}
  i\hbar\pdv{t}\psi(\vec{r}, t) &= \qty[c \vec{\alpha}\vdot \vec{\hat{p}} + \beta m c^2]\psi(\vec{r}, t) \\
  &= \hat{H}\psi(\vec{r}, t)
\end{align*}
where \(\vec{\alpha}\) and \(\beta\) are independent of all derivatives, so commute with all terms except each other.
We also require that
\[ \hat{H}^2 \psi = (c^2\vec{\hat{p}}^2 + m^2 c^4)\psi\]
which after rearranging to compare with the previous ansatz gives us the anticommutator relations
\begin{align*}
  \anticommutator{\alpha^i}{\alpha^j} &= \delta^{ij} & \anticommutator{\alpha^i}{\beta} &= 0
\end{align*}
and note that \(\beta^2 = 1\).
From these requirements, we guess that \(\alpha\) and \(\beta\) must actually be hermitian \(n \times n\) matrices. We also note that they must be traceless, and have eigenvalues of \(\pm 1\).
We use the smallest representation of \(\alpha\) and \(\beta\) which is in four dimensional matrices,
\begin{align*}
  \beta &= \mqty(I&0\\0&-I) & \vec{\alpha}^i &= \mqty(0 & \sigma^i \\ \sigma^i & 0)
\end{align*}
where \(\sigma^i\) are the Pauli matrices,
\begin{align*}
  \sigma^1 &= \mqty(\pmat{1}) & \sigma^2 &= \mqty(\pmat{2}) & \sigma^3 &= \mqty(\pmat{3})
\end{align*}
 which are unitary, hermitian and satisfy the anti-commutation relation required.

 The Hamiltonian is now a 4D matrix, so we must make a wavefunction to match. We give it a \emph{spinor}\footnote{pronounced ``spinner''} which is simply a four component column matrix (not a vector) in the format
 \begin{align*}
   \psi(\vec{r}, t) &= \mqty(\psi_1(\vec{r},t)\\ \psi_2(\vec{r},t)\\ \psi_3(\vec{r},t)\\ \psi_4(\vec{r},t)).
 \end{align*}

\subsection{Continuity}
We go back to the continuity equation, and doing the same thing with the Dirac equation as we did with the others, we find that
\begin{align*}
  \rho &= \psi^\dagger \psi & \vec{j} &= \psi^\dagger \vec{\alpha} \psi & j^\mu &= \mqty(\rho, \vec{j})
\end{align*}
which gives the covariant form of the continuity equation as \[\partial_\mu j^\mu = 0. \]
This is now perfectly well defined for the probability density (huzzah!).

\section{Just Leave It Alone. It's The Only Thing To Do}
Free particle \(\implies \) plane wave, as always.
However, we need a spinor version, so we invent a new spinor, \(w(p)\) which gives us
\begin{align*}
  \psi(\vec{r}, t) &= \exp(-\frac{i}{\hbar}p\vdot x)\,w(p)\\
  &= \exp(-\frac{i}{\hbar}(cp^0 t - \vec{p}\vdot\vec{r}))\,w(p).
\end{align*}
This leads us to a slightly different form of the Dirac equation,
\begin{align}
  p^0 w (p) &= (\vec{\alpha} \vdot \vec{p} + \beta m c)\, w(p)\label{eqn:pspace}
\end{align}
 called the \(p\)-space Dirac equation.

 This can be solved elegantly using a slightly funky formalism where
 \[ w(p) = \mqty(\phi(p)\\\chi(p))\] which is a four-component spinor in terms of two two-component spinors.
 This makes \autoref{eqn:pspace}
 \begin{align*}
   p^0 \mqty(\phi\\\chi) &= \mqty(mc & \vec{\sigma}\vdot \vec{p} \\ \vec{\sigma}\vdot \vec{p} & -mc ) \mqty(\phi\\\chi)
 \end{align*}
 which shows this off as an eigenvalue equation, which is nice, because we know how to solve them.
 Whatever your method of choice, you can find that
 \begin{align*}
   \chi &= \frac{\vec{\sigma}\vdot\vec{p}}{p^0 + mc}\phi\\
   p_0 \phi &= \qty(mc + \frac{(\vec{\sigma}\vdot \vec{p})^2}{p^0 + mc})\phi
 \end{align*}
which apparently shows that
\[(p^0)^2 = m^2c^2 + \vec{p}^2\]
but I'm unconvinced.

\subsection{Can a particle have Feng Shui?}
As usual, we identify \(p^0 = \nicefrac{E}{c}\) with the energy of the particle.
We thus write the positive energy solutions as
\begin{align*}
  w^{(i)}(p) &= \mqty(\phi^{(i)}\\\frac{c \vec{\sigma}\vdot\vec{p} }{E + mc^2}\phi^{(i)}) & \phi^{(1)} &= \mqty(1\\0) & \phi^{(2)} &= \mqty(0\\1)&  i \in { 1, 2 }
\end{align*}
where we have conventionally chosen the spinors (which have to be linearly independent, and preferably orthogonal).

This gives us that
\begin{align*}
  w^{(1)}(p) &= \begin{pmatrix}
    1 \\ 0\\
    \frac{cp^3}{E+mc^2} \\ \frac{c(p^1 + i p^2)}{E+mc^2}
  \end{pmatrix} &
  w^{(2)}(p) &= \begin{pmatrix}
    0\\ 1\\
    \frac{c(p^1 + i p^2)}{E+mc^2} \\ \frac{-cp^3}{E+mc^2}
  \end{pmatrix}
\end{align*}
where it's worth noticing that \( \vec{\sigma}\vdot \vec{p} \) is not a scalar, since \(\vec{\sigma}\) is a horrible abuse of notation for the three Pauli matrices.

\subsection{Negative Energy}
The negative energy solutions (\(p^0 = - \nicefrac{E}{C}\)) are written in terms of \(\chi\), and conventionally we find that
\begin{align*}
  w^{(i)}(-p) &= \mqty(\frac{c \vec{\sigma}\vdot\vec{p} }{E + mc^2}\chi^{(i-2)}\\\chi^{(i-2)}) & \chi^{(1)} &= \mqty(0\\1) & \chi^{(2)} &= \mqty(1\\0) &  i \in { 3, 4 }.
\end{align*}
We choose the negative energy states to have negative momentum because of the Dirac Sea interpretation of particle physics in a faintly vector like object which can be iterated over.

This gives us that
\begin{align*}
  w^{(3)}(-p) &= \begin{pmatrix}
    \frac{c(p^1 + i p^2)}{E+mc^2} \\ \frac{-cp^3}{E+mc^2}\\
    0\\ 1
  \end{pmatrix} &
  w^{(4)}(-p) &= \begin{pmatrix}
    \frac{cp^3}{E+mc^2} \\ \frac{c(p^1 + i p^2)}{E+mc^2}\\
    1 \\ 0
  \end{pmatrix}
\end{align*}
which represent the negative energy states of a spin-\(\nicefrac{1}{2}\) particle.

\section{Let's stop it all moving around, just to keep track}
When \(\vec{p} = 0\), the \(w\) spinors become orthonormal, and we conventionally write them as
\begin{align*}
  w^{(1)} &= \mqty(1\\0\\0\\0) &
  w^{(2)} &= \mqty(0\\1\\0\\0) &
  w^{(3)} &= \mqty(0\\0\\0\\1) &
  w^{(4)} &= \mqty(0\\0\\1\\0)
\end{align*}
and the positive-energy solutions become
\begin{align*}
  \psi^{(i)} &= e^{-i\frac{mc^2}{\hbar}t}\, w^{(i)}
\end{align*}
which means that they are degenerate, and there must be some other operator in the good quantum numbers for the rest frame.

We call this new operator \[\hat{\Sigma}^3 = \smqty(\dmat{1,-1,1,-1}) \] of which the four \(w\) spinors are all eigenvectors\footnote{Eigenspinors? What \emph{is} a spinor? How is it special? This entire section has some very questionable notation.} with values \(\pm 1\).
That this is in fact part of a wider definition that
\begin{align}
  \hat{\Sigma}^i &\equiv \mqty(\dmat{\sigma^i,\sigma^i})
\end{align}
is what leads us to believe that these are spin-\(\nicefrac{1}{2}\) particles.
This comes from us defining \(\hat{s}^i = \nicefrac{1}{2} \hbar \hat{\Sigma}^i \) which is exactly the spin operator.
This only commutes in the rest frame, as does the angular momentum \(\hat{\vec{L}} = \hat{\vec{r}}\times \hat{\vec{p}} \), so neither of these are conserved, or good quantum numbers.
However, the total angular momentum \( \hat{\vec{J}} \) commutes with the Dirac Hamiltonian in all frames.

\subsection{Helicity}
We can define a different operator which also commutes with the Hamiltonian, the Helicity operator
\begin{align*}
  \hat{h}(\vec{p}) &= \mqty(\frac{\vec{\sigma}\vdot \vec{p}}{\abs{\vec{p}}} & 0 \\ 0 & \frac{\vec{\sigma}\vdot \vec{p}}{\abs{\vec{p}}}) \\
  &\propto \hat{\vec{\Sigma}} \vdot \frac{\vec{p}}{\abs{\vec{p}}}
\end{align*}
which has eigenvalues \(\pm 1 \) --- this is the particle's spin along the direction of motion.

\section{The Dirac Sea}
Dirac proposed that all negative energy states are filled, with each energy level holding two electrons --- a spin up and a spin down.
The Pauli exclusion principle means that positive energy electrons can't just transition into a negative energy state without that state already being empty.
The negative energy states are unobservable, and are completely full in a complete vaccuum.

This is also a model for pair production.
The negative energy electron takes \(2(E+mc^2)\) in energy to jump up into its relative positive energy state, leaving behind a `hole' in the sea, which in this model is the positron (discovered after the model).

Quantum field theory fixes the issue that this model doesn't work for bosons.

\section{Covariant Dirac Equation}
The Dirac Equation can be rewritten from
\begin{align*}
  i\hbar \pdv{t} \psi(\vec{r}, t) &= \qty(-i\hbar c \vec{\alpha}\vdot\grad{} + \beta{} mc^2)\,\psi(\vec{r}, t)\intertext{to}
  0 &= \qty[i\hbar\qty(\gamma^0 \pdv{x^0} + \gamma^i \pdv{x^i}) - mc] \psi(\vec{r}, t)
\end{align*}
where we have defined
\begin{align*}
  \gamma &\equiv (\beta, \beta\vec{\alpha}).
\end{align*}

We also now define the Feynman Slash notation, \[\slashed{a} \equiv \gamma^\mu a_\mu \] which is the contraction with the \(\gamma\) spinor.
This allows us to reduce this expression to \[ (i \hbar \slashed{\partial} mc) \psi(x) = 0\] which is really very simple.

The plane wave solutions reduce to \[ \psi(x) = -e^{-\nicefrac{i}{\hbar}p\vdot x} u(p, s) \] where this satisfies
\[ (\slashed{p} - mc)\, u(p, s) = 0 \]
and we've made new, less awkward to write positive energy spinors \[ u(p, s) \equiv w^{(s)}(p).\]


\appendix
\chapter{Gaussian Integrals}
\section{Basic Gaussians}
Lots of Gaussian integrals happen in this subject. In the interests of not getting stuck, here are the results.

The basic Gaussian integral for \( a \in \mathbb{R}^+\) is
\[ \int_\mathbb{R} e^{-ax^2} \dd{x} = \sqrt\frac{\pi}{a} \] which we can then use Feynman's trick of differentiation with respect to the parameter and symmetry arguments to find
\begin{align*}
  \int_\mathbb{R} x^n e^{-ax^2} \dd{x} &= \begin{cases}
    \pdv[\nicefrac{n}{2}]{a} \sqrt\frac{\pi}{a} & \text{if \(n\) even}\\
    0 & \text{otherwise}
  \end{cases}.
\end{align*}

If there is an \(e^{bx} \) power, this may be taken care of using completing the square, given
\begin{align*}
  \int_\mathbb{R} e^{-ax^2+bx} \dd{x} &= \int_\mathbb{R} e^{-a(x - \nicefrac{b}{2a})^2+\nicefrac{b^2}{4a}} \dd{x}\\
  &= \sqrt\frac{\pi}{a} e^{\nicefrac{b^2}{4a}}
\end{align*}
and if we make \(b = ik\) this still works, which shows that the fourier transform of a Gaussian is still a Gaussian.
With these, Feynman's trick still works, and works even better than before as we now have single \(x\) powers to play with --- just keep differentiating.

By closing one's contour cleverly, one can also solve Fresnel (and similar) integrals with a complex integrand. This gives
\begin{align*}
  \int_\mathbb{R} e^{-iax^2+ikx} \dd{x}
  &= \sqrt\frac{i\pi}{a} e^{\nicefrac{-k^2}{4a}}
\end{align*}
which is the same as what you get if you just substitute into the previous result, but it's not obvious that that should be the case.

\section{Nested Gaussians}\label{sec:nested-gaussians}
When it is said that there are a lot of Gaussians, what makes it all the harder is that they all happen \emph{all at once}.

Firstly, we look at a double Gaussian, in the form
\begin{align*}
  I &= \int_{-\infty}^\infty e^{\nicefrac{i}{a} {(x-u)^2}}e^{ \nicefrac{i}{b}(u-y)^2}\dd{u}
\end{align*}
which is a bit tricky because they're coupled.
We use a clever trick, by which we use the translational invariance of these integrals, making a mapping \( u \mapsto u + y \) which makes the whole thing a function of \( z = x - y \).
Utilising this gives us
\begin{align*}
  \frac{i}{a}(z-u)^2 + \frac{i}{b}u^2 &= \frac{i}{a}(z^2 - 2zu + u^2) + \frac{i}{b}u^2\\
  &= i\qty(\frac{1}{a} + \frac{1}{b})\qty(u - \frac{z}{a \qty(\frac{1}{a} + \frac{1}{b})})^2 + \frac{i}{a}z^2 + i \frac{z^2}{a^2 \qty(\frac{1}{a}  + \frac{1}{b})}\\
  &= i\qty(\frac{1}{a} + \frac{1}{b})\qty(u - \frac{z}{ia \qty(\frac{1}{a} + \frac{1}{b})})^2 + iz^2\frac{1}{a + b}
\end{align*}
which with another translation gives us the really nice answer that \[ I = e^{i\frac{z^2}{a + b}} \sqrt{\frac{iab\pi}{a+b}}.\]

The $N$-th nested Gaussian, \( I_N \), is the logical extension of this, and telescopes rather nicely.
We set
\begin{align*}
  I_N &= \idotsint \dd{x_1} \cdots \dd{x_N} \exp(i \qty[ (x_1 - x_0 )^2 + (x_2 - x_1)^2 + \ldots + (x_{N+1} - x_{N})^2])\\
      &= \sqrt{\frac{i\pi}{2}}\idotsint \dd{x_2} \cdots \dd{x_N} \exp(i \qty[ \nicefrac{1}{2}(x_2 - x_0 )^2 + (x_2 - x_1)^2 + \ldots + (x_{N+1} - x_{N})^2])\\
      &= \sqrt{\frac{i\pi}{N+1}}\exp(i \nicefrac{(x_{N+1} - x_0)^2}{N+1})
\end{align*}
which is a nice enough statement.

\printbibliography
\end{document}
