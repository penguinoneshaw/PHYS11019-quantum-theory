\appendix
\chapter{Gaussian Integrals}
\section{Basic Gaussians}
Lots of Gaussian integrals happen in this subject. In the interests of not getting stuck, here are the results.

The basic Gaussian integral for \( a \in \mathbb{R}^+\) is
\[ \int_\mathbb{R} e^{-ax^2} \dd{x} = \sqrt\frac{\pi}{a} \] which we can then use Feynman's trick of differentiation with respect to the parameter and symmetry arguments to find
\begin{align*}
  \int_\mathbb{R} x^n e^{-ax^2} \dd{x} &= \begin{cases}
    \pdv[\nicefrac{n}{2}]{a} \sqrt\frac{\pi}{a} & \text{if \(n\) even}\\
    0 & \text{otherwise}
  \end{cases}.
\end{align*}

If there is an \(e^{bx} \) power, this may be taken care of using completing the square, given
\begin{align*}
  \int_\mathbb{R} e^{-ax^2+bx} \dd{x} &= \int_\mathbb{R} e^{-a(x - \nicefrac{b}{2a})^2+\nicefrac{b^2}{4a}} \dd{x}\\
  &= \sqrt\frac{\pi}{a} e^{\nicefrac{b^2}{4a}}
\end{align*}
and if we make \(b = ik\) this still works, which shows that the fourier transform of a Gaussian is still a Gaussian.
With these, Feynman's trick still works, and works even better than before as we now have single \(x\) powers to play with --- just keep differentiating.

By closing one's contour cleverly, one can also solve Fresnel (and similar) integrals with a complex integrand. This gives
\begin{align*}
  \int_\mathbb{R} e^{-iax^2+ikx} \dd{x}
  &= \sqrt\frac{i\pi}{a} e^{\nicefrac{-k^2}{4a}}
\end{align*}
which is the same as what you get if you just substitute into the previous result, but it's not obvious that that should be the case.

\section{Nested Gaussians}\label{sec:nested-gaussians}
When it is said that there are a lot of Gaussians, what makes it all the harder is that they all happen \emph{all at once}.

Firstly, we look at a double Gaussian, in the form
\begin{align*}
  I &= \int_{-\infty}^\infty e^{\nicefrac{i}{a} {(x-u)^2}}e^{ \nicefrac{i}{b}(u-y)^2}\dd{u}
\end{align*}
which is a bit tricky because they're coupled.
We use a clever trick, by which we use the translational invariance of these integrals, making a mapping \( u \mapsto u + y \) which makes the whole thing a function of \( z = x - y \).
Utilising this gives us
\begin{align*}
  \frac{i}{a}(z-u)^2 + \frac{i}{b}u^2 &= \frac{i}{a}(z^2 - 2zu + u^2) + \frac{i}{b}u^2\\
  &= i\qty(\frac{1}{a} + \frac{1}{b})\qty(u - \frac{z}{a \qty(\frac{1}{a} + \frac{1}{b})})^2 + \frac{i}{a}z^2 + i \frac{z^2}{a^2 \qty(\frac{1}{a}  + \frac{1}{b})}\\
  &= i\qty(\frac{1}{a} + \frac{1}{b})\qty(u - \frac{z}{ia \qty(\frac{1}{a} + \frac{1}{b})})^2 + iz^2\frac{1}{a + b}
\end{align*}
which with another translation gives us the really nice answer that \[ I = e^{i\frac{z^2}{a + b}} \sqrt{\frac{iab\pi}{a+b}}.\]

The $N$-th nested Gaussian, \( I_N \), is the logical extension of this, and telescopes rather nicely.
We set
\begin{align*}
  I_N &= \idotsint \dd{x_1} \cdots \dd{x_N} e^{i \qty[ (x_1 - x_0 )^2 + (x_2 - x_1)^2 + \ldots + (x_{N+1} - x_{N})^2]}\\
      &= \sqrt{\frac{i\pi}{2}}\idotsint \dd{x_2} \cdots \dd{x_N} \exp[i \qty[ \nicefrac{1}{2}(x_2 - x_0 )^2 + (x_2 - x_1)^2 + \ldots + (x_{N+1} - x_{N})^2]]\\
      &= \sqrt{\frac{i\pi}{N+1}}\exp[i \nicefrac{(x_{N+1} - x_0)^2}{N+1}]
\end{align*}
which is a nice enough statement.

\chapter{Functional Derivatives}
A functional derivative generalises the derivative from
\begin{align*}
  \pdv{J_i}\sum_j J_j x_j &= x_i & \pdv{J_i} &= \delta_{ij}
\end{align*}
to cover a smooth function on a continuous domain such that
\begin{align*}
  \pdv{J(t)}J(t^\prime) &= \delta(t-t^\prime) & \fdv{J(t)}  \int_{t_a}^{t_b} J(t^\prime) x(t^\prime) \dd{t^\prime} &= \int_{t_a}^{t_b} \delta(t - t^\prime) x(t^\prime) \dd{t^\prime}\\
  &&&= \begin{cases}
    x(t) & t_a < t^\prime < t_b\\
    0 & \text{otherwise}.
  \end{cases}
\end{align*}

Functional derivatives have the following properties:
\begin{description}
  \item [Linearity] \[ \fdv{(\lambda F + \mu G)[\rho]}{\rho(x)} = \lambda \fdv{F[\rho]}{\rho(x)} + \mu \fdv{G[\rho]}{\rho(x)}\]
  for \(\lambda, \mu\) constant,
  \item [Chain Rule] \[ \fdv{(FG)[\rho]}{\rho} = \fdv{F[\rho]}{\rho(x)}G[\rho] + F[\rho]\fdv{F[\rho]}{\rho(x)} \]
  \item [Product Rules] These are logical. Look them up if you need them.
\end{description}

We can use an equivalent of the Euler-Lagrange equation to find the functional derivative, \[ \fdv{F}{\rho(r)} = \pdv{f}{\rho} - \partial^\mu \pdv{f}{\qty(\partial^\mu \rho)}.\]
